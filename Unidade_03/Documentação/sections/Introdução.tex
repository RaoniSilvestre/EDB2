\section{Introdução}

As árvores B utilizam a característica de armazenar múltiplas chaves 
em cada nó para organizar eficientemente os dados e os ponteiros, permitindo 
que operações como busca, inserção e remoção sejam realizadas de forma rápida. 
Além disso, sua construção garante que todas as folhas estejam sempre no mesmo nível, 
assegurando balanceamento e consistência na estrutura.


\subsection{Propriedades}

Seja \( d \) um número natural. Uma árvore B de ordem \( d \) é uma árvore ordenada 
que pode ser vazia ou satisfazer as seguintes condições:
\begin{itemize}
    \item \textbf{A raiz}: É uma folha ou possui, no mínimo, dois filhos.
    \item \textbf{Nós internos}: Cada nó, exceto a raiz e as folhas, possui pelo menos \( d + 1 \) filhos.
    \item \textbf{Capacidade máxima}: Cada nó pode conter, no máximo, \( 2d + 1 \) filhos.
    \item \textbf{Nivelamento das folhas}: Todas as folhas estão no mesmo nível, como mencionado anteriormente.
\end{itemize}

\subsection{Estrutura}

Para a construção da árvore B em Rust, definimos a seguinte estrutura:

\subsubsection{\texttt{BTree}}

A estrutura \texttt{BTree} representa a árvore principal e é definida como:

\begin{lstlisting}
    pub struct BTree {
        root: Node,
        grau: i32,
    }
\end{lstlisting}

Os campos de \texttt{BTree} são:
\begin{itemize}
    \item \textbf{\texttt{root}}: Representa a raiz da árvore B. É o ponto inicial para 
    todas as operações, como busca, inserção e remoção.
    \item \textbf{\texttt{grau}}: Define o grau da árvore B. O grau \( d \) determina a 
    capacidade mínima e máxima dos nós.
\end{itemize}

\subsubsection{\texttt{Node}}

A estrutura \texttt{Node} representa os nós da árvore B e é definida como:

\begin{lstlisting}
    pub struct Node {
        keys: Vec<Key>,
        children: Vec<Node>,
        is_leaf: bool,
        grade: i32,
    }
\end{lstlisting}

Os campos de \texttt{Node} são:
\begin{itemize}
    \item \textbf{\texttt{keys}}: Um vetor que armazena as chaves presentes no nó. 
    As chaves são mantidas ordenadas para facilitar as operações de busca.
    \item \textbf{\texttt{children}}: Um vetor que contém os nós filhos, representando 
    a hierarquia da árvore. Para nós folha, este vetor estará vazio.
    \item \textbf{\texttt{is\_leaf}}: Um campo booleano que indica se o nó é uma folha. 
    Nós folha não possuem filhos.
    \item \textbf{\texttt{grade}}: Representa o grau do nó, que pode ser usado para verificar 
    o número atual de chaves ou filhos presentes.
\end{itemize}

\subsubsection{\texttt{Key}}

A estrutura \texttt{Key} define as informações armazenadas em cada nó e é definida como:

\begin{lstlisting}
    pub struct Key {
        key: i32,
        nome: String,
        quantidade: usize,
    }
\end{lstlisting}

Os campos de \texttt{Key} são:
\begin{itemize}
    \item \textbf{\texttt{key}}: A chave propriamente dita, utilizada para ordenação e busca na árvore.
    \item \textbf{\texttt{nome}}: Um identificador associado à chave, que pode armazenar 
    informações descritivas ou metadados.
    \item \textbf{\texttt{quantidade}}: Representa a quantidade associada à chave.
\end{itemize}
