\section{Busca}

O algoritmo de busca por uma chave \( x \) em uma árvore B é semelhante ao utilizado em árvores binárias 
de busca. No entanto, a principal diferença reside no fato de que, em uma árvore B, cada nó pode conter 
múltiplas chaves, tornando necessário percorrer todas as chaves presentes em um nó antes de decidir 
em qual subárvore continuar a busca. 

A implementação em rust ficou da seguinte forma:

\begin{lstlisting}
    pub fn find(&self, k: Key) -> Option<Key> {
        let mut curr_node = &self.root;
        loop {
            match curr_node.search(&k) {
                SearchResult::Find(i) => return curr_node.key(i).cloned(),
                SearchResult::GoDown(i) => match curr_node.child(i) {
                    None => return None,
                    Some(next_node) => curr_node = next_node,
                },
            }
        }
    }
\end{lstlisting}

\begin{itemize}
    \item A função busca a chave \( k \) começando pela raiz.
    \item Cada nó pode ter várias chaves, então search é 
    usado para decidir se a chave está no nó ou em qual subárvore buscar.
    \item Se a chave for encontrada, é retornada uma cópia dela.
    \item Se não for encontrada e não houver mais filhos para explorar, a função retorna None.
\end{itemize}
---

Na implementação da função search, temos:

\begin{lstlisting}
    pub fn search(&self, k: &Key) -> SearchResult {
        for (i, key) in self.keys.iter().enumerate() {
            match key.key.cmp(&k.key) {
                Ordering::Less => {}
                Ordering::Equal => return SearchResult::Find(i),
                Ordering::Greater => return SearchResult::GoDown(i),
            }
        }
        SearchResult::GoDown(self.keys.len())
    }
\end{lstlisting}

\begin{itemize}
    \item Retorna $Find(i)$ se a chave \( k \) for encontrada no índice \( i \). 
    \item Retorna $GoDown(i)$ para indicar que a busca deve continuar no filho \( i \).
\end{itemize}
---

Esse método mantém a eficiência da busca na árvore B, garantindo um tempo de execução 
$O(logn)$, já que a altura da árvore é logarítmica em relação ao número de chaves armazenadas.
