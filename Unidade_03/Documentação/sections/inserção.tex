\section{Inserção}

O processo de inserção em uma Árvore B segue os seguintes passos principais:

\begin{enumerate}
    \item **Localizar o local correto para inserir a chave:** utiliza a mesma lógica do método \texttt{search}.
    \item **Inserir e balancear a árvore:** verifica se o nó é folha ou interno:
    \begin{itemize}
        \item Se for folha, a chave é inserida diretamente.
        \item Se o nó estiver cheio, ele é dividido usando a função \texttt{split}, e a chave do meio sobe para o nó pai.
        \item Caso a raiz precise ser dividida, cria-se uma nova raiz, aumentando a altura da árvore.
    \end{itemize}
\end{enumerate}

---

\subsection{Método \texttt{insert} na \texttt{BTree}}

O método \texttt{insert} na \texttt{BTree} realiza a inserção na raiz da árvore, verificando se há necessidade de criar uma nova raiz.

\begin{lstlisting}
pub fn insert(&mut self, k: Key) {
    let insertion = self.root.insert(k);

    if let InsertionResult::AddToFater(k, new_node) = insertion {
        *self = Self::new_root(self.root.clone(), k, new_node);
    }
}
\end{lstlisting}

\begin{itemize}
    \item Primeiro, chama o método \texttt{insert} no nó raiz (\texttt{self.root}).
    \item Caso a raiz seja dividida (\texttt{AddToFater}), uma nova raiz é criada para manter a árvore balanceada.
\end{itemize}

Se a altura da árvore precisar aumentar, o método \texttt{new\_root} cria uma nova raiz com dois filhos.

---

\subsection{Método \texttt{insert} no \texttt{Node}}

O método \texttt{insert} no \texttt{Node} realiza a inserção recursiva nos nós, seguindo as etapas abaixo:

\begin{lstlisting}
pub fn insert(&mut self, k: Key) -> InsertionResult {
    let result = self.search(&k);

    if self.is_leaf {
        return self.try_insert(k);
    }

    if let SearchResult::GoDown(i) = result {
        let child = self.children[i].insert(k);

        if let InsertionResult::AddToFater(middle_key, new_node) = child {
            self.children.insert(i + 1, new_node);
            self.children.sort();
            self.keys.insert(i, middle_key);
            self.keys.sort();

            if self.is_full() {
                return self.split_node();
            }
        }
    }

    InsertionResult::Inserted
}
\end{lstlisting}

\begin{itemize}
    \item Busca o local correto para a inserção da chave usando o método \texttt{search}.
    \item Se o nó for folha, insere diretamente utilizando o método \texttt{try\_insert}.
    \item Caso contrário, chama \texttt{insert} recursivamente no nó filho correspondente.
    \item Após a inserção no filho:
    \begin{itemize}
        \item Caso o nó filho seja dividido (\texttt{AddToFater}), insere o novo nó filho e a chave do meio no nó atual.
        \item Se o nó atual estiver cheio, chama \texttt{split\_node} para dividi-lo.
    \end{itemize}
\end{itemize}

---

\subsection{Função \texttt{split\_node}}

O método \texttt{split\_node} realiza a divisão de um nó quando ele está cheio, movendo a chave do meio para o nó pai. A implementação segue:

\begin{lstlisting}
pub fn split_node(&mut self) -> InsertionResult {
    let t = self.grade as usize;
    let mid = t;

    let mut right_node = Node::new(self.is_leaf, self.grade);

    right_node.keys = self.keys.drain(mid + 1..).collect();

    if !self.is_leaf {
        right_node.children = self.children.drain(mid + 1..).collect();
    }

    let middle_key = self.keys.remove(mid);

    InsertionResult::AddToFater(middle_key, right_node)
}
\end{lstlisting}

O método segue as etapas:

\begin{itemize}
    \item Define \texttt{mid} como o índice central do nó.
    \item Cria um novo nó à direita (\texttt{right\_node}).
    \item Move as chaves à direita do índice \texttt{mid} para \texttt{right\_node}.
    \item Se o nó não for folha, também move os filhos correspondentes para \texttt{right\_node}.
    \item Remove a chave do meio, que será enviada para o nó pai.
    \item Retorna \texttt{AddToFater}, indicando a necessidade de inserir a chave do meio no pai.
\end{itemize}

--- 
