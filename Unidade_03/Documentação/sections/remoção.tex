\section{Remoção}

A remoção de uma chave \( x \) em uma Árvore B pode ser tratada de diferentes formas, 
dependendo da posição da chave dentro da árvore. Os principais casos são:

\begin{itemize}
    \item Se \( x \) está em um nó folha: a chave é simplesmente removida do nó.
    \item Se \( x \) está em um nó interno: \( x \) é substituída pela chave \( y \), que é 
    a maior chave do subárvore esquerda (ou, alternativamente, a menor chave da subárvore direita).
\end{itemize}

No entanto, ao remover uma chave, o número de chaves em um nó pode ficar menor que \( d \), 
violando as propriedades da Árvore B. Para corrigir essa violação, dois métodos podem ser aplicados:

\begin{itemize}
    \item **Redistribuição**: as chaves são redistribuídas entre nós irmãos para manter o número 
    mínimo de chaves em cada nó.
    \item **Concatenação (ou fusão)**: se a redistribuição não for possível, dois nós são mesclados 
    em um único nó, reduzindo a altura da árvore, se necessário.
\end{itemize}

---

\subsection{Redistribuição de Chaves}

Sejam \( P \) e \( Q \) dois nós irmãos adjacentes que, juntos, possuem pelo menos \( 2d \) chaves. 
Seja \( W \) o nó pai de \( P \) e \( Q \). O processo de redistribuição funciona da seguinte forma:

\begin{itemize}
    \item Uma chave do pai \( W \) é movida para o nó que ficou com menos chaves.
    \item Uma chave do nó irmão adjacente é promovida para o pai \( W \) para manter a estrutura balanceada.
\end{itemize}

Se a redistribuição não for possível (ou seja, se \( P \) e \( Q \) juntos possuem menos que \( 2d \) chaves), então a concatenação é aplicada.

No seguinte fluxograma é possível visualizar essa estrategia:

\begin{figure}[h]
    \centering
    \includegraphics[width=1.2\textwidth]{imagens/_fluxograma.png}
    \caption{Legenda da imagem}
    \label{fig:exemplo}
\end{figure}
