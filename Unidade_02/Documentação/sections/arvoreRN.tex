\section{Árvore Rubro Negra}

A seção atual detalha a implementação completa de uma árvore rubro-negra, desenvolvida na linguagem C, incluindo suas principais operações: inserção, rotação, remoção, busca.

\vspace{3mm}

\subsection{Estrutura de Dados}

A estrutura de dados utilizada para representar um nó de uma árvore rubro-negra é apresentada a seguir:

\begin{lstlisting}
    typedef struct arvore_rn_t {
        int chave;                 
        cor_t cor;                 
        struct arvore_rn_t *esq;   
        struct arvore_rn_t *dir;   
        struct arvore_rn_t *pai;   
    } arvore_rn_t;
\end{lstlisting}

O campo \texttt{chave} armazena o valor do nó, enquanto \texttt{cor} indica se o nó é vermelho ou preto. o tipo \texttt{cor\_t} é uma enumeração composta pelos itens \texttt{RUBRO} e \texttt{NEGRO}, nesta mesma ordem. Os ponteiros \texttt{esq}, \texttt{dir} e \texttt{pai} apontam para os filhos esquerdo e direito, e para o pai do nó, respectivamente.

\vspace{3mm}

\subsection{Principais Funções de Manipulação}

\subsubsection{Inserção}

A inserção em árvores rubro-negras é uma operação que mantém as propriedades da árvore balanceada após adicionar um novo nó. Inicialmente, o novo nó é sempre inserido como vermelho para evitar violações da altura negra. Em seguida, ajustes são realizados para corrigir possíveis violações nas propriedades da árvore, utilizando rotações e mudanças de cores. O processo garante que a complexidade de busca permaneça \(O(\log n)\), mantendo a eficiência do balanceamento dinâmico.

Inicialmente a função \texttt{criar\_no()} é chamada, ela recebe um inteiro como parâmetro e aloca dinâmicamente um novo nó para ser adicionado à uma árvore rubro-negra. Além disso, ela define os ponteiros e a cor do nó com valores padrões.

\begin{lstlisting}
    arvore_rn_t *criar_no(int chave) {
        arvore_rn_t *novo_no = (arvore_rn_t *)malloc(sizeof(arvore_rn_t));
        novo_no->chave = chave;
        novo_no->cor = RUBRO;
        novo_no->pai = NULL;
        novo_no->esq = NULL;
        novo_no->dir = NULL;
        return novo_no;
    }
\end{lstlisting}

\vspace{3mm}

Após isso, chamamos a função \texttt{inserir()}, que recebe a raiz da árvore e um novo nó, e o insere na árvore, de forma recursiva, com base somente nas proprieades de uma árvore binária de busca.

\begin{lstlisting}
    arvore_rn_t *inserir(arvore_rn_t *arvore, arvore_rn_t *no) {
        if (arvore == NULL) {
            return no;
        } else if (no->chave < arvore->chave) {
            arvore->esq = inserir(arvore->esq, no);
            arvore->esq->pai = arvore;
        } else if (no->chave > arvore->chave) {
            arvore->dir = inserir(arvore->dir, no);
            arvore->dir->pai = arvore;
        }

        return arvore;
    }
\end{lstlisting}

\vspace{3mm}

\subsubsection{Balanceamento e proteção das propriedades}

Após a inserção de um novo nó em uma árvore rubro-negra, precisamos garantir que as suas propriedades sejam mantidas. Para isso, implementamos as funções \texttt{rotacao\_esq()} e \texttt{rotacao\_dir} para rotacionar a árvore e a função \texttt{corrigir\_balanceamento()} para corrigir erros durante a inserção simples de um nó, mantendo assim as propriedades da árvore rubro-negra.

Abaixo, as funções \texttt{rotacao\_esq()} e \texttt{rotacao\_dir} que recebem um nó da árvore como parametro e rotacionam a partir desse nó.

\begin{lstlisting}
    void rotacao_esq(arvore_rn_t *x) {
        arvore_rn_t *y = x->dir;
        x->dir = y->esq;
        if (y->esq != NULL)
            y->esq->pai = x;

        y->pai = x->pai;
        if (x->pai == NULL)
            raiz = y;
        else if (x == x->pai->esq)
            x->pai->esq = y;
        else
            x->pai->dir = y;

        y->esq = x;
        x->pai = y;
    }
\end{lstlisting}

\begin{lstlisting}
    void rotacao_dir(arvore_rn_t *x) {
        arvore_rn_t *y = x->esq;
        x->esq = y->dir;
        if (y->dir != NULL)
            y->dir->pai = x;

        y->pai = x->pai;
        if (x->pai == NULL)
            raiz = y;
        else if (x == x->pai->dir)
            x->pai->dir = y;
        else
            x->pai->esq = y;

        y->dir = x;
        x->pai = y;
    }
\end{lstlisting}

\vspace{3mm}

A função \texttt{corrigir\_balanceamento()} recebe um nó como parâmetro e é chamada após a inserção de todo novo nó na árvore rubro-negra para manter as propriedades e balanceamento da árvore, ela segue iterativamente os casos possíveis após a inserção de um nó, para garantir que as propriedades da árvore se mantenham e o nó esteja onde deveria estar.

\begin{lstlisting}
    void corrigir_balanceamento(arvore_rn_t *no) {
        arvore_rn_t *pai = NULL;
        arvore_rn_t *avo = NULL;

        while (no != raiz && no->cor == RUBRO && no->pai->cor == RUBRO) {
            pai = no->pai;
            avo = pai->pai;

            if (pai == avo->esq) {
                arvore_rn_t *tio = avo->dir;

                if (tio != NULL && tio->cor == RUBRO) {
                    avo->cor = RUBRO;
                    pai->cor = NEGRO;
                    tio->cor = NEGRO;
                    no = avo;
                } else {
                    if (no == pai->dir) {
                        rotacao_esq(pai);
                        no = pai;
                        pai = no->pai;
                    }
                    rotacao_dir(avo);
                    cor_t temp = pai->cor;
                    pai->cor = avo->cor;
                    avo->cor = temp;
                    no = pai;
                }
            } else {
                arvore_rn_t *tio = avo->esq;

                if (tio != NULL && tio->cor == RUBRO) {
                    avo->cor = RUBRO;
                    pai->cor = NEGRO;
                    tio->cor = NEGRO;
                    no = avo;
                } else {
                    if (no == pai->esq) {
                        rotacao_dir(pai);
                        no = pai;
                        pai = no->pai;
                    }
                    rotacao_esq(avo);
                    cor_t temp = pai->cor;
                    pai->cor = avo->cor;
                    avo->cor = temp;
                    no = pai;
                }
            }
        }

        raiz->cor = NEGRO;
    }
\end{lstlisting}

\vspace{3mm}

\subsubsection{Remoção}

A função \texttt{remover()}, recebe a árvore ou sub-árvore de onde o nó deve ser removido e uma chave de busca. Se não exisistia previamente um nó com a chave desejada, ela retorna a árvore original, caso contrário, ela remove o nó e rebalanceia a árvore nova, antes de retorná-la.

\begin{lstlisting}
    arvore_rn_t *remover(arvore_rn_t *arvore, int chave) {
        if (arvore == NULL) return NULL;

        if (chave < arvore->chave) {
            arvore->esq = remover(arvore->esq, chave);
        } else if (chave > arvore->chave) {
            arvore->dir = remover(arvore->dir, chave);
        } else {
            if (arvore->esq == NULL) {
                arvore_rn_t *temp = arvore->dir;
                free(arvore);
                return temp;
            } else if (arvore->dir == NULL) {
                arvore_rn_t *temp = arvore->esq;
                free(arvore);
                return temp;
            }

            arvore_rn_t *rightMin = find_min(arvore->dir);
            arvore->chave = rightMin->chave;
            arvore->dir = remover(arvore->dir, rightMin->chave);
        }
        corrigir_balanceamento(arvore);
        return arvore;
    }
\end{lstlisting}

\vspace{3mm}

\subsection{Principais Funções de consulta}

\subsubsection{Busca}

A busca em uma árvore rubro-negra é igual à busca em uma árvore binária de busca, a função \texttt{buscar()} recebe a árvore ou sub-árvore onde a busca deve ocorrer e um valor para ser buscado. Se o valor for achado, ela retorna um ponteiro para o nó onde o valor está armazenado, caso contrário, ela retorna \texttt{NULL}.

\begin{lstlisting}
    arvore_rn_t *buscar(arvore_rn_t *arvore, int chave) {
        if (arvore == NULL) {
            return 0;
        }

        if (chave < arvore->chave) {
            return buscar(arvore->esq, chave);
        }

        if (chave > arvore->chave) {
            return buscar(arvore->dir, chave);
        }

        if (chave == arvore->chave) {
            return arvore;
        }
        return NULL;
    }
\end{lstlisting}

\vspace{3mm}

\subsubsection{Visualização em ordem}

Na visualização em ordem, utilizando a função \texttt{inorder()}, os nós da árvore rubro-negra são visitados seguindo a sequência: filho da esquerda, raiz e, por fim, filho da direita. Esse tipo de travessia resulta em uma listagem dos elementos em ordem crescente.

\begin{lstlisting}
    void inorder(arvore_rn_t *arvore) {
        if (arvore != NULL) {
            inorder(arvore->esq);
            printf("%d ", arvore->chave);
            inorder(arvore->dir);
        }
    }
\end{lstlisting}

\vspace{3mm}

\subsubsection{Visualização em Nível}

Na visualização em nível, a árvore completa é impressa no terminal, deitada, de forma que a primeira coluna da saída seja a raíz, a segunda coluna sejam seus filhos, a terceira coluna sejam seus netos, e assim por diante. Na função \texttt{imprimir\_arvore()}, é recebido um ponteiro para a raíz da árvore e essa função chama a função \texttt{imprimir\_subarvore()}, que realiza a impressão de forma recursiva de todos os nós da árvore rubro-negra, de forma que os nós pretos são impressos na cor branca com o fundo preto e os nós vermelhos são impressos na cor branca com o fundo vermelho.

\begin{lstlisting}
    void imprimir_arvore(arvore_rn_t *raiz) {
        printf("Visualização da árvore deitada:\n");
        imprimir_sub_arvore(raiz, 0);
    }

    void imprimir_sub_arvore(arvore_rn_t *arvore, int espacos) {
        if (arvore != NULL) {
            espacos += 5;
            imprimir_sub_arvore(arvore->dir, espacos);
            printf("\n");
            for (int i = 5; i < espacos; i++) {
                printf(" ");
            }
            if (VERMELHO(arvore)) {
                printf("\e[0;41m");
            } else if (PRETO(arvore)) {
                printf("\e[0;40m");
            }
            printf("%d\e[0m", arvore->chave);
            imprimir_sub_arvore(arvore->esq, espacos);
        }
    }
\end{lstlisting}

\vspace{3mm}

Abaixo, um exemplo de saída (sem a visualização colorida):

\begin{lstlisting}
    Visualização da árvore deitada:

                    15
            10
                9
        8
            6
    5
            4
                3
        2
            1
\end{lstlisting}

\vspace{3}