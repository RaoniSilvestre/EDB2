\section{Árvore AVL}

Esta seção apresenta a implementação completa de uma Árvore AVL, esta a seguir, desenvolvida na linguagem C,
incluindo suas principais operações: inserção, remoção, balanceamento, busca. 
Além disso, para fins de visualização, também foi implementada uma funcionalidade de exportação para formato DOT.
A Árvore AVL é uma árvore binária de busca auto-balanceada que garante operações eficientes, como inserções e buscas, mantendo a complexidade \(O(\log n)\).

\subsection{Estrutura de Dados}

A estrutura de dados da Árvore AVL (\texttt{arvore\_avl}) é definida com os seguintes atributos:

\begin{verbatim}
typedef struct arvore_avl {
    int valor;                // Valor armazenado no nó
    int altura_esq;           // Altura da subárvore esquerda
    int altura_dir;           // Altura da subárvore direita
    struct arvore_avl *esq;   // Ponteiro para o filho esquerdo
    struct arvore_avl *dir;   // Ponteiro para o filho direito
} arvore_avl;
\end{verbatim}

O campo \texttt{valor} armazena o dado do nó, enquanto \texttt{altura\_esq} e \texttt{altura\_dir} mantêm a altura de cada subárvore. 
Isso facilita o balanceamento da árvore, que é realizado após cada inserção ou remoção.

\subsection{Funções Principais da Árvore AVL}

\subsubsection{Cálculo da Altura}

A função \texttt{calcular\_altura} retorna a altura da subárvore. Caso o nó seja nulo (\texttt{NULL}), a altura será considerada 0. 
Este cálculo é essencial para determinar se a árvore precisa de balanceamento.

\begin{verbatim}
int calcular_altura(arvore_avl *arv) {
    if (arv == NULL) {
        return 0;
    }
    int altura_esq = calcular_altura(arv->esq);
    int altura_dir = calcular_altura(arv->dir);
    return (altura_esq > altura_dir ? altura_esq : altura_dir) + 1;
}
\end{verbatim}

\subsubsection{Criação de um Novo Nó}

A função \texttt{criar\_novo\_no} aloca memória para um novo nó da árvore e inicializa seus campos.

\begin{verbatim}
arvore_avl *criar_novo_no(int valor) {
    arvore_avl *novo_no = (arvore_avl *)malloc(sizeof(arvore_avl));
    if (novo_no == NULL) {
        fprintf(stderr, "Erro ao alocar memória para o nó.\n");
        exit(1);
    }
    novo_no->valor = valor;
    novo_no->altura_esq = 0;
    novo_no->altura_dir = 0;
    novo_no->esq = NULL;
    novo_no->dir = NULL;
    return novo_no;
}
\end{verbatim}

\subsubsection{Rotações de Balanceamento}

O balanceamento da Árvore AVL é garantido por meio de rotações,
que reorganizam os nós para manter a diferença de altura entre as subárvores esquerda e direita dentro do limite de 1.

\paragraph{Rotação à Esquerda}  
A rotação à esquerda reorganiza os nós em torno do filho direito do nó desbalanceado.

\begin{verbatim}
arvore_avl *rotacao_esquerda(arvore_avl *raiz) {
    if (raiz == NULL || raiz->dir == NULL) {
        return raiz;
    }
      
    arvore_avl *novo_raiz = raiz->dir;
    arvore_avl *subarvore_dir = novo_raiz->esq;
      
    novo_raiz->esq = raiz;
    raiz->dir = subarvore_dir;
      
    raiz->altura_esq = calcular_altura(raiz->esq);
    raiz->altura_dir = calcular_altura(raiz->dir);
      
    novo_raiz->altura_esq = calcular_altura(novo_raiz->esq);
    novo_raiz->altura_dir = calcular_altura(novo_raiz->dir);
      
    return novo_raiz;
}
\end{verbatim}

\paragraph{Rotação à Direita}  
A rotação à direita é o análogo da rotação à esquerda, mas reorganiza os nós em torno do filho esquerdo.

\begin{verbatim}
arvore_avl *rotacao_direita(arvore_avl *raiz) {
    if (raiz == NULL || raiz->esq == NULL) {
        return raiz;
    }
      
    arvore_avl *novo_raiz = raiz->esq;
    arvore_avl *subarvore_esq = novo_raiz->dir;
      
    novo_raiz->dir = raiz;
    raiz->esq = subarvore_esq;
      
    raiz->altura_esq = calcular_altura(raiz->esq);
    raiz->altura_dir = calcular_altura(raiz->dir);
      
    novo_raiz->altura_esq = calcular_altura(novo_raiz->esq);
    novo_raiz->altura_dir = calcular_altura(novo_raiz->dir);
      
    return novo_raiz;
}
\end{verbatim}

\subsubsection{Inserção de um Nó}

A função \texttt{inserir\_no} insere um valor na árvore de maneira recursiva. Após cada inserção, a árvore é balanceada.

\begin{verbatim}
arvore_avl *inserir_no(arvore_avl *raiz, int valor) {
    arvore_avl *novo_no;
      
    if (raiz == NULL) {
        return criar_novo_no(valor);
    }
      
    if (valor < raiz->valor) {
        raiz->esq = inserir_no(raiz->esq, valor); 
    } else if (valor > raiz->valor) {
        raiz->dir = inserir_no(raiz->dir, valor); 
    }
      
    raiz->altura_esq = calcular_altura(raiz->esq);
    raiz->altura_dir = calcular_altura(raiz->dir);
      
      
    raiz = balancear_arvore(raiz);
    return raiz;
}
\end{verbatim}

\subsubsection{Remoção de um Nó}

A função \texttt{remover\_no} remove um valor da árvore e a rebalanceia, quando necessário.

\begin{verbatim}
arvore_avl *remover_no(arvore_avl *raiz, int valor) {
    if (raiz == NULL) {
        return NULL;
    }
      
    if (valor < raiz->valor) {
        raiz->esq = remover_no(raiz->esq, valor);
    } else if (valor > raiz->valor) {
        raiz->dir = remover_no(raiz->dir, valor);
    } else {
        if (raiz->esq == NULL && raiz->dir == NULL) {
        free(raiz);
        return NULL;
        } else if (raiz->esq == NULL) {
        arvore_avl *aux = raiz->dir;
        free(raiz);
        return aux;
        } else if (raiz->dir == NULL) {
        arvore_avl *aux = raiz->esq;
        free(raiz);
        return aux;
        } else {
        arvore_avl *aux = raiz->dir;
        while (aux->esq != NULL) {
            aux = aux->esq;
        }
        raiz->valor = aux->valor;
        raiz->dir = remover_no(raiz->dir, aux->valor);
        }
    }
      
    raiz->altura_esq = calcular_altura(raiz->esq);
    raiz->altura_dir = calcular_altura(raiz->dir);
  
    return balancear_arvore(raiz);
}
\end{verbatim}

\subsubsection{Busca em uma Árvore AVL}

A função \texttt{buscar} localiza um valor na árvore. Caso o valor não seja encontrado, retorna \texttt{NULL}.

\begin{verbatim}
arvore_avl *buscar(arvore_avl *raiz, int valor) {

    if (raiz == NULL) {
      return 0;
    }
  
    if (valor < raiz->valor) {
      return buscar(raiz->esq, valor);
    }
  
    if (valor > raiz->valor) {
      return buscar(raiz->dir, valor);
    }
  
    if (valor == raiz->valor) {
      return raiz;
    }
    return NULL;
}
\end{verbatim}

\subsubsection{Exportação para DOT}

A função \texttt{exportar\_para\_dot}, em conjunto com a função \texttt{gerar\_dot}, gera um arquivo no formato DOT, possibilitando a visualização da árvore em ferramentas gráficas.
Embora essa funcionalidade não seja uma característica intrínseca de uma Árvore AVL,
foi incluída neste código para tornar sua análise mais acessível e intuitiva, facilitando a compreensão de sua estrutura e funcionamento.

\begin{verbatim}
void exportar_para_dot(arvore_avl *raiz, const char *nome_arquivo) {
    FILE *arquivo = fopen(nome_arquivo, "w");
    if (arquivo == NULL) {
        fprintf(stderr, "Erro ao abrir o arquivo %s\n", nome_arquivo);
        exit(1);
    }

    fprintf(arquivo, "digraph G {\n");
    gerar_dot(arquivo, raiz);
    fprintf(arquivo, "}\n");
    fclose(arquivo);
}
\end{verbatim}


\subsection{Conclusão sobre a arvore AVL}

A Árvore AVL é uma estrutura de dados solida e eficiente para manipulação de informações. 
Ela assegura que as operações de busca, inserção e remoção sejam realizadas em tempo 
O(logn), graças ao balanceamento automático proporcionado pelos cálculos de altura e rotações. 
Essa característica permite que a árvore permaneça equilibrada, independentemente da ordem em que os nós são inseridos ou removidos. 
Por sua eficiência, as Árvores AVL têm ampla aplicação em sistemas que exigem alto desempenho,
como bancos de dados, sistemas de arquivos e em outras diversas áreas da computação.