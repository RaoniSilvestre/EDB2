\section{Heap}

Nessa seção, são explicadas as estruturas de MinHeap e MaxHeap. Ambas foram implementadas em
Rust, devido as atividades da primeira unidade, pois as ordenações da primeira únidade foram feitas
em Rust. Primeiramente explicaremos a Max Heap e após isso, mais resumidamente devido a semelhança, 
a Min Heap.

\subsection{Max Heap}

A Max Heap é uma espécie de árvore, que é comumente implementada como lista. A única 
condição para que uma árvore/lista seja considerada uma Max Heap, é que para todo nó, 
seu filhos devem ter prioridade menor que o pai.

Desse modo, nota-se que o maior elemento da Max Heap sempre será a raiz dela
(ou o H[0], no caso da lista).

\subsubsection{Estrutura}

Na implementação da Max Heap, criamos um WrapperType(uma classe), para encapsular o
funcionamento da Max Heap:

\vspace{3mm}

\begin{lstlisting}
pub struct MaxHeap<T> {
    data: Vec<T>,
}
\end{lstlisting}

\vspace{3mm}

\subsubsection{Alteração de prioridade}

Para realizar a alteração de prioridade na heap(o tira casaco, bota casaco dela),
é preciso implementar as funções de subir e descer na heap. Elas servem para manter a
principal propriedade da (max)heap: cada nó tem prioridade maior que seus filhos.

\vspace{3mm}

\textbf{1. Função subir} 

\vspace{3mm}

Para a função de subir(bubble\_up), a implementação é simples.
Pegamos a heap(\&mut self) e a posição que ira subir como argumentos.
Devido as propriedades da heap, sabemos que o pai da $self[i]$ está na
posição $i/2$, e dessa forma verificamos se o filho tem prioridade maior que o pai.
Se for o caso, as posições do filho e do pai são trocadas,
e então chama-se a função recursivamente na posição do pai.


\begin{lstlisting}
pub fn bubble_up(&mut self, mut index: usize) {
    while index > 0 {
        let parent = (index - 1) / 2;
        if self[index] <= self[parent] {
            break;
        }
        self.swap(index, parent);
        index = parent;
    }
}
\end{lstlisting}

\vspace{3mm}

\textbf{2. Função descer} 

\vspace{3mm}

Para a função de descer, é um pouco mais complicado. Visto que cada item da heap terá 2 filhos,
é preciso levar em conta os dois, para decidir o que fazer no algoritmo.


Primeiramente, pegamos a quantidade de elementos na Heap e então fazemos uma iteração.

Para cada iteração, comparamos a prioridade do index atual com a prioridade de seus filhos,
caso algum dos filhos seja maior que o pai, realizamos o swap do pai com o filho, e
repetimos o processo.
Se nenhum dos filhos é maior que o pai, significa que o item desceu até a posição correta,
e paramos o loop.

\begin{lstlisting}
pub fn bubble_down(&mut self, mut index: usize) {
    let last_index = match self.len() {
        0 => 0,
        n => n - 1,
    };
    loop {
        let left_child = (2 * index) + 1;
        let right_child = (2 * index) + 2;
        let mut largest = index;
        if left_child <= last_index && self[left_child] > self[largest] {
            largest = left_child;
        }
        if right_child <= last_index && self[right_child] > self[largest] {
            largest = right_child;
        }
        if largest == index {
            break;
        }
        self.swap(index, largest);
        index = largest;
    }
}
\end{lstlisting}

\subsubsection{Inserção}

Adiciona-se o valor ao final da lista e então usa a função subir(bubble\_up)
para corrigir a prioridade do novo valor inserido.

\begin{lstlisting}
pub fn push(&mut self, value: T) {
    self.data.push(value);
    self.bubble_up(self.data.len() - 1);
}
\end{lstlisting}

\subsubsection{Remoção}

Se a lista está vazia, apenas retorna $None$.

Se não, troca-se a posição do último elemento com o primeiro,
retira-se o novo último da lista e utiliza-se a função descer(bubble\_down)
no novo primeiro elemento.

\begin{lstlisting}
pub fn pop(&mut self) -> Option<T> {
    if self.is_empty() {
        return None;
    }

    let last_index = self.len() - 1;
    self.swap(0, last_index);
    let max_value = self.data.pop();
    self.bubble_down(0);
    max_value
}
\end{lstlisting}

\subsubsection{Construção da MaxHeap}

Para a construção de uma MaxHeap a partir de uma lista,
foi utilizada a trait (interface/typeclass)
From\texttt{<}Vec\texttt{<}T\texttt{>}\texttt{>}.
Essa trait é o método padrão para fazer "mapeamento de tipos",
ou seja, mapear um tipo A em um tipo B. Nesse caso,
mapeamos o tipo Vec\texttt{<}T\texttt{>} em MaxHeap\texttt{<}T\texttt{>}.

Quanto ao algoritmo, ele se baseia em considerar que as folhas da heap já
são heaps. Com isso, temos que a partir da posição $len/2 + 1$, a lista já é uma heap.

Desse modo, só o que falta é organizar a MaxHeap de $0$ até $len/2$. Para isso, usamos
a função descer(bubble\_down) de $len/2$ até $0$ e então retornamos a heap.


\begin{lstlisting}
impl<T: Default + Ord> From<Vec<T>> for MaxHeap<T> {
    fn from(data: Vec<T>) -> Self {
        let mut heap = MaxHeap { data };
        let len = heap.len();
        for i in (0..len / 2).rev() {
            heap.bubble_down(i);
        }
        heap
    }
}
\end{lstlisting}

\subsubsection{(Max)Heapsort}

No heapsort, o que fazemos é consumir a MaxHeap para retornar um Vetor ordenado.

Para isso, criamos um vetor mutável com o tamanho da MaxHeap, e como o primeiro elemento
da MaxHeap é sempre o maior dela, se você retirar repetidamente os valores da heap e inseri-los na lista, você
terá uma lista ordenada descendente. Quando a MaxHeap é esvaziada, temos uma lista ordenada
"ao contrário", por isso usamos a função reverse retornar uma lista ordenada ascendente.

\begin{lstlisting}
pub fn heapsort(mut self) -> Vec<T> {
    let mut sorted = Vec::with_capacity(self.len());
    while let Some(max) = self.pop() { sorted.push(max); }
    sorted.reverse();
    sorted
}
\end{lstlisting}

\subsection{Min Heap}

Em uma Min Heap, a regra a ser seguida é o contrário da Max Heap: Para todo nó,
o os filhos dele devem ter prioridade MAIOR que ele, ou seja, a prioridade do pai
é sempre MENOR que seus filhos. 
Desse modo, a implementação da Min Heap
é muito similar, a diferença é que é necessário "flipar" as condicionais nas funções de subir
e descer. 

Além disso, no (min)heapsort, não será necessário fazer um .reverse() na lista, pois os elementos 
já estarão em ordem crescente devido a propriedade do Min Heap.
